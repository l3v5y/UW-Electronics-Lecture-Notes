\documentclass[11pt]{article} % use larger type; default would be 10pt
\usepackage{amsmath}
\usepackage{tikz}
\usepackage{geometry} % to change the page dimensions
\geometry{a4paper} % or letterpaper (US) or a5paper or....
\geometry{margin=1in}

\usepackage{fancyhdr} % This should be set AFTER setting up the page geometry
\pagestyle{fancy} % options: empty , plain , fancy
\renewcommand{\headrulewidth}{0pt} % customise the layout...
\lhead{ES3C5}\chead{Lecture 1}\rhead{07/01/2013}
\lfoot{}\cfoot{\thepage}\rfoot{Olly Levett}

\begin{document}

\section{Signal}
	\begin{itemize}
	\item A quantity that can be varied to convey information
	\item Converted into electrical form using a transducer
	\item e.g. sine waves
		
	\end{itemize}

	\begin{figure}[h]
		\begin{tikzpicture}
			\draw (0,0) -- (1,0) -- (1,1) -- (2,1) --(2,0) -- (3,0);
		\end{tikzpicture}
		\centering
		\caption{A square wave for some reason}
	\end{figure}

\subsection{Laplace transforms (LT)}
	\begin{itemize}
	\item For modelling a linear sstem sing a transfer function
	\item LT of function $f(t)$ in time domain is
		\begin{equation}
			F(s) = \int^\infty _0 f(t)e^{-st}dt =\mathcal{L}\{f(t)\}
		\end{equation}
		with laplace variable $s = \sigma j\omega$ with dimension $time^{-1}$
	\end{itemize}

	Example 1.1
	\begin{equation}
		f(t) =  e^{\alpha t} 
	\end{equation}
	\begin{eqnarray}
		F(s) &=& \int^\infty_0 e^{\alpha t} e^{-st} dt = \mathcal{L}\{f(t)\} \nonumber \\
		&=& \int^\infty_0 e^{-(s-\alpha)t} dt \nonumber \\
		&=& -\frac{1}{s-\alpha}\left[{e^{-(s-\alpha)t}}\right]_0^\infty \nonumber \\
		&=& \frac{1}{s-\alpha}
	\end{eqnarray}

\subsection{Inverse LT}
	\begin{itemize}
		\item $\mathcal{L}^{-1} F(s)=f(t)$
		\item F(s) and  f(t) are LT pairs
		\item Obtained using partial fraction method and table of LT pairs
	\end{itemize}

	Example 1.2

	Determine the signal given
		\begin{eqnarray}
			F(s) &=& \frac{s+4}{s(s+2)} \nonumber \\
			&&\mbox{using partial fraction method} \nonumber \\
			F(s) &=& \frac{2}{s} - \frac{1}{s+2} \nonumber \\
			&&\mbox{from databook table 1.1 2nd \& 4th rows} \nonumber \\
			f(t)&=&2u(t)-e^{-2t}
		\end{eqnarray}
		for $t\ge$ 0, where $u(t)$ is a unit step

	\begin{figure}[h]
		\begin{tikzpicture}
			\draw (0,0) -- (1,0)  -- (1,1) node [left] {$1$} -- (2,1) -- (4,1);
			\draw[->] (0,0) -> (4,0) node [below right] {$x$};
		\end{tikzpicture}
		\centering
		\caption{}
	\end{figure}

\subsection{Properties of LT}
	\begin{description}
		\item[Property 1] if $x(t)\leftrightarrow X(s)$ and $y(t)\leftrightarrow Y(s)$ then $x(t)+-y(t) \leftrightarrow X(s)+- Y(s)$
		\item[Property 2] if $x(t)\leftrightarrow X(s)$  and $K$ is constant, then $Kx(t)\leftrightarrow KX(s)$

		Example 1.3 

		Determine LT of $v(t) = 3cos4t$
		From table 1.1 7th row
			\begin{equation}
				\mathcal{L}\{cos\omega t\} = \frac{s}{s^2+\omega^2}
			\end{equation}
		i.e., $\omega=4$, and using property 2 gives
			\begin{equation}
				 V(s) =  \frac{3s}{s^2+16}
			\end{equation}
		\item[Property 3] Derivatives
			\begin{equation}
				\mathcal{L}\left\{\frac{d^nf(t)}{dt^n}\right\} = s^n  F(s) -s^{n-1}f(0)-s^{n-2}f^1(0)  ...  f^{n-1}(0)
			\end{equation}
			where $f^n(t)$ denotes the n\textsuperscript{th} derivative of f(t)
		Assume quiescent state, i.e. all system variables and their derivatives are 0 at $t=0$,
			\begin{equation}
				\mathcal{L}\left\{\frac{d^nf(t)}{dt^n}\right\}  = s^nF(s)
			\end{equation}
		valid assumption in all practical systems (no power$\rightarrow$off)

		Example 1.4
			
			Given $$\tau \frac{dy(t)}{dt}+y(t) = kx(t)$$  where $x(t)$ and $y(t)$ are input and output of a system respectively.
			\begin{equation}
				\tau s Y(s) + Y(s) = kX(s) \\
				Y(s) = X(s)\left[\frac{k1}{1+s^\tau }\right]
			\end{equation}

		\item[Property 4] Integration
			\begin{equation}
				\int^t_0f(t)dt\leftrightarrow \frac{F(s)}{s}
			\end{equation}

		\item[Property 5] Time-shift (delay)
			\begin{equation}
				\mathcal{L}\left\{f(t-T)\right\}=e^{-sT}F(s)
			\end{equation}
		\item[Property 6]
			If $\mathcal{L}\{f(t)\}=F(s)$
			then $\mathcal{L}\{e^{at}f(t)\} = F(s-a)$
	\end{description}
\end{document}