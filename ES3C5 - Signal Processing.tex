\documentclass[11pt]{article} % use larger type; default would be 10pt
\usepackage{amsmath}
\usepackage{tikz}
\usepackage{geometry} % to change the page dimensions
\geometry{a4paper} % or letterpaper (US) or a5paper or....
\geometry{margin=1in}
\usepackage{circuitikz}

\usepackage{amsmath}
\usepackage{amsfonts}
\usepackage{amssymb}
\usepackage{fancyhdr} % This should be set AFTER setting up the page geometry
\pagestyle{fancy} % options: empty , plain , fancy
\renewcommand{\headrulewidth}{0pt} % customise the layout...
\lhead{ES3C5}\chead{}\rhead{Signal Processing}
\lfoot{}\cfoot{\thepage}\rfoot{}


\usepackage{amsthm}

\theoremstyle{definition}
\newtheorem{example}{Example}[subsection]

\begin{document}

\section{Signals}
	\begin{itemize}
	\item A quantity that can be varied to convey information
	\item Converted into electrical form using a transducer
	\item e.g. sine waves
		
	\end{itemize}

	\begin{figure}[h]
		\begin{tikzpicture}
			\draw (0,0) -- (1,0) -- (1,1) -- (2,1) --(2,0) -- (3,0);
		\end{tikzpicture}
		\centering
		\caption{A square wave for some reason}
	\end{figure}

\subsection{Laplace transforms (LT)}
	\begin{itemize}
	\item For modelling a linear sstem sing a transfer function
	\item LT of function $f(t)$ in time domain is
		\begin{equation}
			F(s) = \int^\infty _0 f(t)e^{-st}dt =\mathcal{L}\{f(t)\}
		\end{equation}
		with laplace variable $s = \sigma j\omega$ with dimension $time^{-1}$
	\end{itemize}

\begin{example}
	\begin{equation}
		f(t) =  e^{\alpha t} 
	\end{equation}
	\begin{eqnarray}
		F(s) &=& \int^\infty_0 e^{\alpha t} e^{-st} dt = \mathcal{L}\{f(t)\} \nonumber \\
		&=& \int^\infty_0 e^{-(s-\alpha)t} dt \nonumber \\
		&=& -\frac{1}{s-\alpha}\left[{e^{-(s-\alpha)t}}\right]_0^\infty \nonumber \\
		&=& \frac{1}{s-\alpha}
	\end{eqnarray}
\end{example}
\subsection{Inverse LT}
	\begin{itemize}
		\item $\mathcal{L}^{-1} F(s)=f(t)$
		\item F(s) and  f(t) are LT pairs
		\item Obtained using partial fraction method and table of LT pairs
	\end{itemize}

\begin{example}
	Determine the signal given
		\begin{eqnarray}
			F(s) &=& \frac{s+4}{s(s+2)} \nonumber \\
			&&\mbox{using partial fraction method} \nonumber \\
			F(s) &=& \frac{2}{s} - \frac{1}{s+2} \nonumber \\
			&&\mbox{from databook table 1.1 2nd \& 4th rows} \nonumber \\
			f(t)&=&2u(t)-e^{-2t}
		\end{eqnarray}
		for $t\ge$ 0, where $u(t)$ is a unit step

	\begin{figure}[h]
		\begin{tikzpicture}
			\draw (0,0) -- (1,0)  -- (1,1) node [left] {$1$} -- (2,1) -- (4,1);
			\draw[->] (0,0) -> (4,0) node [below right] {$x$};
		\end{tikzpicture}
		\centering
		\caption{}
	\end{figure}
\end{example}

\subsection{Properties of LT}
	\begin{description}
		\item[Property 1] if $x(t)\leftrightarrow X(s)$ and $y(t)\leftrightarrow Y(s)$ then $x(t)+-y(t) \leftrightarrow X(s)+- Y(s)$
		\item[Property 2] if $x(t)\leftrightarrow X(s)$  and $K$ is constant, then $Kx(t)\leftrightarrow KX(s)$

\begin{example}
		Determine LT of $v(t) = 3cos4t$
		From table 1.1 7th row
			\begin{equation}
				\mathcal{L}\{cos\omega t\} = \frac{s}{s^2+\omega^2}
			\end{equation}
		i.e., $\omega=4$, and using property 2 gives
			\begin{equation}
				 V(s) =  \frac{3s}{s^2+16}
			\end{equation}
		\item[Property 3] Derivatives
			\begin{equation}
				\mathcal{L}\left\{\frac{d^nf(t)}{dt^n}\right\} = s^n  F(s) -s^{n-1}f(0)-s^{n-2}f^1(0)  ...  f^{n-1}(0)
			\end{equation}
			where $f^n(t)$ denotes the n\textsuperscript{th} derivative of f(t)
		Assume quiescent state, i.e. all system variables and their derivatives are 0 at $t=0$,
			\begin{equation}
				\mathcal{L}\left\{\frac{d^nf(t)}{dt^n}\right\}  = s^nF(s)
			\end{equation}
		valid assumption in all practical systems (no power$\rightarrow$off)
\end{example}
\begin{example}	
			Given $$\tau \frac{dy(t)}{dt}+y(t) = kx(t)$$  where $x(t)$ and $y(t)$ are input and output of a system respectively.
			\begin{equation}
				\tau s Y(s) + Y(s) = kX(s) \\
				Y(s) = X(s)\left[\frac{k1}{1+s^\tau }\right]
			\end{equation}
\end{example}
		\item[Property 4] Integration
			\begin{equation}
				\int^t_0f(t)dt\leftrightarrow \frac{F(s)}{s}
			\end{equation}

		\item[Property 5] Time-shift (delay)
			\begin{equation}
				\mathcal{L}\left\{f(t-T)\right\}=e^{-sT}F(s)
			\end{equation}
		\item[Property 6]
			If $\mathcal{L}\{f(t)\}=F(s)$
			then $\mathcal{L}\{e^{at}f(t)\} = F(s-a)$
	\end{description}



\section{Laplace transfer function (TF)}
\begin{itemize}
\item For a linear and stationary system
\begin{equation}
TF = \frac{\mathcal{L}\{output\}}{\mathcal{L}\{input\}}
\end{equation}
ie $\mathcal{L}\{output\} = TF x \mathcal{L}\{input\}$

with all intitial conditions assymed zero.
\item TF describes the dynamics of the system

\end{itemize}
A linear system obeys the principle of superposition, i.e. if $x_1 \rightarrow y_1$ and $x_2\rightarrow y_2$ then $x_1+x_2 \rightarrow y_1 + y_2$ where $x_n$ and $y_n$ are respectively the input and output of the sytem.

\subsection{Resistors}
\begin{figure}[h]

\begin{circuitikz}
	\draw
	(0,2) 
	to[short,o-*] (2,2) 
	node[anchor=west,above right]{$i(t)$}
	to[R, l=$R$,*-*] (2,0)
	to[short,*-o] (0,0);
	\draw[->] (0,0.3) --(0,1) node[left]{$v(t)$} --  (0,1.7);
\end{circuitikz}
\centering
\caption{Simple resistor network}
\end{figure}

\begin{equation}
v(t) = Ri(t)
\end{equation}

taking LT and assume zero initial conditions 
\begin{equation}
V(s) =RI(s)
\end{equation}
\subsection{Capacitors}

\begin{figure}[h]
\begin{circuitikz}
	\draw
	(0,2)
	to[short,o-*] (2,2) 
	node[anchor=west,above right]{$i(t)$}
	to[C, l=$C$,*-*] (2,0)
	to[short,*-o] (0,0);
	\draw[->] (0,0.3) --(0,1) node[left]{$v(t)$} --  (0,1.7);

\end{circuitikz}
\centering
\caption{Simple capacitor network}
\end{figure}
\begin{equation}
v(t) = \frac{1}{C}\int i(t) dt
\end{equation}
Taking LT and assume zero initial conditions
\begin{equation}
V(s) = \frac{I(s)}{sC}
\end{equation}

\subsection{Inductors}

\begin{figure}[h]
\begin{circuitikz}
	\draw
	(0,2)
	to[short,o-*] (2,2) 
	node[anchor=west,above right]{$i(t)$}
	to[L, l=$L$,*-*] (2,0)
	to[short,*-o] (0,0);
	\draw[->] (0,0.3) --(0,1) node[left]{$v(t)$} --  (0,1.7);
\end{circuitikz}
\centering
\caption{Simple inductor network}
\end{figure}
\begin{equation}
v(t) = L\frac{di(t)}{dt}
\end{equation}
Take LT and assume zero initial conditions
\begin{equation}
V(s) = sLI(s)
\end{equation}

\subsection{Kirchoff's Laws}
\begin{enumerate}
\item The total current flowing towards a node is equal to the total current flowing from that node
\item In a closed circuit, the algebraic sum of the products of the current and the resistance of each part of the circuit is equal to the resultant e.m.f. in the circuit. 

Alternatively, in a given loop, the sum of voltage rises is equal to the sum of voltage drops.

\end{enumerate}
The TF of a system can be found by finding the LT of each componenent and applying Kirchoff's laws.

\begin{example}
	See Fig. \ref{ex:int}
	\begin{figure}[h]
		\begin{circuitikz}
			\draw 	(0,2)
			to[R,l=$R$,o-*] (3,2) 
			to[C, l=$C$,*-*] (3,0)
			to[short,*-o] (0,0);

			\draw (3,2) to[short,*-o ](5,2);
			\draw (3,0) to[short,*-o] (5,0);

			\draw[->] (0,0.3) --(0,1) node[left]{$v_i(t)$} --  (0,1.7);
			\draw[->] (5,0.3) --(5,1) node[right]{$v_o(t)$} --  (5,1.7);
		\end{circuitikz}
		\centering
		\caption{An integrating circuit}
		\label{ex:int}
	\end{figure}

	\begin{equation}
		V_0(s)=\frac{I(s)}{sC}
	\end{equation}
From 2nd law,
	\begin{equation}
		V_i(s)=RI(s)+\frac{I(s)}{sC}
	\end{equation}
TF
	\begin{eqnarray}
		\frac{V_o}{V_i}&=&\frac{\frac{I(s)}{sC}}{RI(s)+\frac{I(s)}{sC}} \nonumber\\
		&=& \frac{1}{sRC+1}
	\end{eqnarray}
\end{example}
\begin{example}
	See Fig. \ref{ex:electnet}
	\begin{figure}[h]
		\begin{circuitikz}
			\draw 	(0,2)
			to[R,l=$R_1$,o-*] (3,2) 
			to[C, l=$C$,*-*] (3,0)
			to[short,*-o] (0,0);
			\draw (3,2) 
			to[L,*-*] (6,2)
			to[R, l=$R_2$,*-*] (6,0)
			to[short,*-*] (3,0);

			\draw (6,2) to[short,*-o ](8,2);
			\draw (6,0) to[short,*-o] (8,0);


			\draw[<-] (2.5,0.3) --(2.5,1) node[left]{$i_1(t)$} --  (2.5,1.7);
			\draw[<-] (5.5,0.3) --(5.5,1) node[left]{$i_2(t)$} --  (5.5,1.7);

			\draw[->] (0,0.3) --(0,1) node[left]{$v_i(t)$} --  (0,1.7);
			\draw[->] (8,0.3) --(8,1) node[right]{$v_o(t)$} --  (8,1.7);
		\end{circuitikz}
		\centering
		\caption{An example electrical network}
		\label{ex:electnet}
	\end{figure}

Applying 2nd law to 1st loop,
\begin{equation}
V_i(s)=R_1I_1(s)+\frac{I_1(s)}{sC}-\frac{I_2(s)}{sC}
\end{equation}
Applying 2nd law to 2nd loop,
\begin{equation}
\frac{I_2(s)}{sC} - \frac{I_1(s)}{sC} + sLI_2(s)+R_2I_2(s)=0
\end{equation}
\end{example} 
Solving simultaneously,
\begin{equation}
V_i(s)=I_2(s)\left( s^2LCR_1+s\left[ CR_1R_2+L\right] + \left[R_1+R_2\right]\right)
\end{equation}
note
\begin{equation}
V_o(S)=R_2I_2(s)
\end{equation}
$\therefore$ TF is
\begin{equation}
\frac{V_o}{V_i} = \frac{R_2}{s^2LCR_1+s\left[ CR_1R_2+L\right] + \left[R_1+R_2\right]}
\end{equation}

\section{Test signals and dynamic response}
\subsection{Unit step input}
	\begin{figure}[h]
		\begin{tikzpicture}
			\draw (0,0) -- (1,0) node[below]{$0$} -- (1,1) node [left] {$1$} -- (2,1) -- (4,1) node[right]{$x(t)$};
			\draw[->] (0,0) -- (4,0) node[below right]{$x$};
		\end{tikzpicture}
		\centering
		\caption{Step response}
	\end{figure}
\begin{equation}
X(s)=\frac{1}{s}
\end{equation}
\begin{example}
Given $H(s)=\frac{1}{1+\tau s}$ (a servo)
\begin{eqnarray}
Y(s) &=& \frac{1}{1+\tau s} \times \frac{1}{s} \nonumber \\
&=& \frac{\frac{1}{\tau}}{s\left( s+ \frac{1}{\tau}\right)}
\end{eqnarray}

\begin{figure}[h]
	\centering
	\caption{Dynamic response of a servo when subject to a unit step}
\end{figure}

\end{example}


\subsection{Unit ramp input}
\begin{figure}[h]
	\centering
	\caption{Dynamic response of a servo when subject to a unit step}
\end{figure}
Where $\tan {(\theta)} = 1$ ie
\begin{eqnarray}
x(t) &=& t \nonumber \\
X(s) &=& \frac{1}{s^2}
\end{eqnarray}

\begin{example}

Given $H(s) = \frac{1}{1+\tau s}$
\begin{eqnarray}
Y(s) &=& \frac{1}{1+\tau s} \times \frac{1}{s^2}  \nonumber  \\
&=& \frac{\tau}{s+\frac{1}{\tau} - \frac{\tau}{s} + \frac{1}{s^2}}
\end{eqnarray}

Thus,
\begin{eqnarray}
y(t) &=& \tau e^{-\frac{t}{\tau}} - \tau + t \nonumber \\
&=& t-\tau\left(1-e^{-\frac{t}{\tau}}\right)
\end{eqnarray}
\end{example}
\end{document}