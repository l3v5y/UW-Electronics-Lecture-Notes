\documentclass[a4paper,twocolumn]{article}

\usepackage[utf8]{inputenc} % set input encoding (not needed with XeLaTeX)

\usepackage{amsmath}

\usepackage[margin=0.8in]{geometry}

\geometry{a4paper}

\ifxetex
  \usepackage{fontspec}
  \usepackage{xunicode}
  \defaultfontfeatures{Mapping=tex-text} % To support LaTeX quoting style
  \setmainfont{Cambria}
\else
\fi

\usepackage{graphicx}
\graphicspath{{gfx/}}

\usepackage{booktabs} 
\usepackage{array}
\usepackage{paralist}
\usepackage{verbatim}
\usepackage{caption}
\usepackage{subcaption}
\usepackage{subfloat}
\usepackage{url}
\usepackage{lscape}
\usepackage{amsmath}
\usepackage{fancyhdr}
\usepackage{pgfplots}
\usepackage{siunitx}
\usepackage{tikz}
\usetikzlibrary{positioning,decorations.pathreplacing,shapes,patterns}
\usepackage[europeanresistors,americaninductors,siunitx]{circuitikz}
\usepackage{amssymb}
\pagestyle{fancy}
\renewcommand{\headrulewidth}{0pt}
\lhead{}\chead{\date{}}\rhead{1006511}


\usepackage[hidelinks]{hyperref}
\usepackage[sort&compress,capitalise,noabbrev]{cleveref}

\usepackage{tabularx}
\newcolumntype{R}{>{\raggedleft\arraybackslash}X}%

\usepackage{setspace}
\singlespacing

% Some macros to make re-spacing a little easier
\usepackage{setspace}
% set to singlespace for normal and doublespace for submission
\newcommand{\customspacing}{doublespace}
\newcommand{\bs}{\begin{\customspacing}}
\newcommand{\es}{\end{\customspacing}}

\let\Contentsline\contentsline
\renewcommand\contentsline[3]{\Contentsline{#1}{#2}{#3}}

\usepackage{url}
\usepackage{eurosym}

\title{ES335 \\ Communication Systems \\ MSL}
\author{1006511\\
		Oliver Levett\\
		%School of Engineering,\\
  		%University of Warwick,\\
  		%Coventry,\\
  		%United Kingdom,\\
  		%CV4 7AL\\
  		\href{mailto:o.levett@warwick.ac.uk}{o.levett@warwick.ac.uk}}

\newcommand{\infint}{ \int_{-\infty}^{\infty}}
\begin{document}
	\section{Part 1}
		\subsection{Motivation}
			How long does it take to double anything
			\begin{equation}
				\left( 1+\frac{x}{100}\right)^n = 2
			\end{equation}
			Where $x$ is the growth rate and $n$ is the number of years.
			\begin{equation}
				n \approx \frac{70}{x}
			\end{equation}

		\subsection{Fundamental Concepts}
			For a probability distribution
			\begin{equation}
				\infint f(x) dx =1
			\end{equation}
			Expectation $E[X]$ and variance $\sigma^2$ are given in the databook.

			\subsubsection{Fourier Transforms}
				Fourier Transforms (F) changes between time domain and frequency domain
				\begin{align}
					X(f)= F \left\{ x(t) \right\} & = \infint x(t)e^{-j 2 \pi f t} dt\\
					x(t) = F ^{-1}\left\{ X(f) \right\} & = \infint X(f)e^{j 2 \pi f t} dt
				\end{align}

			\subsubsection{Delta function}
				Infinitesimally brief
				\begin{equation}
					\delta (x-a)=a \qquad x\ne a
				\end{equation}
				Normalised
				\begin{equation}
					\infint 	\delta (x-a) dx = 1
				\end{equation}
				Sifting
				\begin{equation}
					\infint f(x) \delta(x-a)dx =f(a)
				\end{equation}
				Fourier transform of delta function
				\begin{equation}
					\infint \delta (f \pm f_0) e^{\pm j 2 \pi f \tau} df = e^{\pm j 2 \pi f_0 \tau}
				\end{equation}

			\subsubsection{Transfer function}
				The FT of an impulse response is the transfer function.
				For an RC circuit the impulse  is given by \cref{eq:rc_ir}.
				\begin{equation}
					h(t) = \frac{1}{RC}e^{-\frac{t}{RC}} \qquad t\ge 0 \label{eq:rc_ir}
				\end{equation}
				The transfer function is given by \cref{eq:rc_tf}
				\begin{equation}
					H(t) = \frac{1}{1+j2\pi f RC}\label{eq:rc_tf}
				\end{equation}
			\subsubsection{Autocorrelation}
				Measure of signal similarity at different times
				\begin{align}
					R_{ff}(\tau) &= \infint f(t+\tau)f^\ast(t) d\tau \\
					R_{ff}(\tau) &= \infint f(t)f^\ast(t-\tau) d\tau
				\end{align}
				Where $f^\ast$ represents the complex conjugate (real function, $f\ast = f$).
				Also,
				\begin{equation}
					R_{xx} = E \left\{ x(t)x(t-\tau) \right\}=E \left\{x(t+\tau)x(t)\right\}
				\end{equation}

				Can be written as $R_x(\tau)$
			\subsubsection{Signal Energey and Power}
				The energy of a signal is given by
				\begin{equation}
					E = \infint \left| f(t)\right|^2 dt \qquad \mathrm{Joules}
				\end{equation}
				which is a slight problem for infinitely long signals, so practaically,
				\begin{equation}
					P =\lim_{T \to \infty} \frac{1}{T} \int^{\frac{T}{2}}_{-\frac{T}{2}} 
					\left| f(t)\right|^2 dt \qquad \mathrm{Watts}
				\end{equation}
				The DC power is given from the autocorrelation function
				\begin{equation}
					\lim_{\tau \to \pm\infty}\left\{ R_{x}(\tau) \right\}
				\end{equation}
				and the mean power is given as
				\[R_x(0)\]

			\subsubsection{Parseval's Theorem}
				\begin{equation}
					E =  \infint \left| f(t)\right|^2 dt =
					\infint \underbrace{ \left| F(t)\right|^2 }_{\text{Energy Density}} df 
					\qquad \mathrm{Joules}
				\end{equation}
				or
				\begin{equation}
					P =\lim_{T \to \infty} \frac{1}{T} \int^{\frac{T}{2}}_{-\frac{T}{2}} 
					\left| f(t)\right|^2 dt =
					\infint \underbrace{ \left| S(f)\right| }_{\text{PSD}} df \qquad
					\mathrm{Watts}
				\end{equation}
			\subsubsection{Wiener-Kinchine Theorem}
				For random signals, the autocorelation function and PSD form a Fourier transform pair
					\begin{align}
						P(f) &= F(R(\tau)) \\
						R(\tau) &= F^{-1}(P(f))
					\end{align}
		\subsection{Digitisation}
			Sample is quantised into multiple steps
			\subsubsection{Sampling}
			SQER is given by
				\begin{equation}
					sqer = \frac{3N^2}{2}
				\end{equation}
			Where N is the number of steps between limits.

			Highest frequency measurable is signal rate divided by 2 word length
	\section{Part 2}
		\subsection{Baseband Pulse Transmission}
			\subsubsection{Gaussian Noise Model}
				\begin{equation}
					p(v) \frac{1}{\sigma \sqrt{2 \pi}}
					\exp\left( -\frac{(v - \overline{v})^2}{2\sigma^2}\right)
				\end{equation}
				For thermal noise, zero mean, so $\overline{v} =0$. PSD is constant, usually
				$\frac{N_0}{2}$

	\section{Part 3}
		\subsection{Modulated Data Transmission}
			\subsubsection{Amplitude Shift Keying}
				ASK, also known as OOK is very simple, but susceptable to noise.
				If the signal amplitude is small, SNR is poor.
				Tends to be avoided.
				\begin{equation}
					f(t) = \begin{cases}
					A_cos(2\pi f_ct) & 0\le t < T \quad \text{`'1'}\\
								0 & 0\le t < T \quad \text{`'0'}
								\end{cases}
				\end{equation}
				with T the bit time, and $\frac{1}{T}$ the signalling rate.
				The bandwidth is $\frac{2}{T}$.
				\begin{equation}
					P_e = \frac{1}{2} erfc \left[ \frac{A_c}{2 \sqrt{2} \sigma}\right]
				\end{equation}
				The energy per bit $E_b$ is half $E_d$, given by
				\begin{equation}
					P_e = \frac{1}{2} erfc \left[ \frac{E-b}{2 N_0}\right]
				\end{equation}

			\subsubsection{Frequency Shift Keying}
				Easy to implement, but bandwidth heavy. Used in low data rate situations
				\begin{equation}
					f(t) = \begin{cases}
						A_c\cos(2\pi f_1t) & 0\le t < T \quad \text{`'1'}\\
						A_c\cos(2\pi f_0t) & 0\le t < T \quad \text{`'0'}
								\end{cases}
				\end{equation}
				PSD is
				\begin{equation}
					\frac{1}{T} + \frac{1}{T}+f_1-f_0 =
						2B_t\left( 1+ \frac{\delta f}{2 B_T} \right)
				\end{equation}
				Can be detected with two BPFs, one at $f_0$ and one at $f_1$.
				\begin{equation}
					P_e = \frac{1}{2} erfc \left[ \frac{A_c}{2\sigma}\right]
				\end{equation}
				The energy per bit $E_b$ is half $E_d$, given by
				\begin{equation}
					P_e = \frac{1}{2} erfc \left[ \frac{E-b}{2 N_0}\right]
				\end{equation}

			\subsubsection{Phase Shift Keying}
				Same bandwidth as ASK, but best error performance
				\begin{equation}
					f(t) = \begin{cases}
						A_c\cos(2\pi f_ct) & 0\le t < T \quad \text{`'1'}\\
						-A_c\cos(2\pi f_ct) & 0\le t < T \quad \text{`'0'}
								\end{cases}
				\end{equation}
				\begin{equation}
					P_e = \frac{1}{2} erfc \left[ \frac{A_c}{\sqrt{2}\sigma}\right]
				\end{equation}
				Bandwidth is $2B_T$ like ASK.

				The energy per bit $E_b$ is given as
				\begin{equation}
					P_e = \frac{1}{2} erfc \left[ \frac{E-b}{N_0}\right]
				\end{equation}

		To increase throughput with a limited bandwidth, use multiple levels. Can implement  encoding schemes.

		Constellation diagrams are used.

		PSK can be extended to multiple phase shifts, such as \ang{45} rather than \ang{180}.

		QAM uses phase and amplitude to achieve multiple levels per transmission. Uses phase and amplitude.
	\section{Part 4}
		\subsection{ECC}
			Achieve detection and correction of errors by adding extra data and redundancy.

			To achieve increased throughput (data rate must be trebled for 3x redundancy), BER goes up gives a disadvantage.

			\subsubsection{Parity}
				Using a block check byte or in conjunction with per-word parity, can detect and correct errors

			\subsubsection{Linear Block Codes}
				Uses addition (XOR, so no carry) and multiplcation (AND).

				For a souce of M messages of length k ($M=2^k$), then with $r$ redundant bits, the message length is $2^{k+r}$. Bit rate is increaesed to $\frac{k+r}{k}$ times the uncoded rate - original message is still present. $n=k+r$

				Hamming distance $d_{min}$ is the minimum difference between any pair of codes in a given set.

				Given $D$ errors in recieved codeword, if we can detect with certainty that the received codeword is not valid when
				\begin{equation}
					D \le d_{min} -1
				\end{equation}
				 $t$ erors can be detected when
				 \begin{equation}
				 	2t+1 \le d_{min} \le 2t+2
				 \end{equation}
				 and
				 \begin{equation}
				 	d_{min} \le n-k+r
				 \end{equation}

				 Generators

				 \begin{equation}
				 	\mathbf{A} = \begin{bmatrix} a_1 & a_2 & \ldots & a_k \end{bmatrix}
				 \end{equation}
				 encode via a generator matrix
				 \begin{equation}
				 	\mathbf{T} = \mathbf{A} \cdot \mathbf{G}
				 \end{equation}
				 For (7,4),
				 \begin{equation}
				 	\mathbf{T} = \begin{bmatrix} a_1 & a_2 & a_3& a_4 & c_1& c_2& c_3\end{bmatrix}
				 \end{equation}


\section{Part 5}
\subsection{Data Compression and Cryptography}
\end{document}