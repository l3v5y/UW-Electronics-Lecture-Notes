\documentclass[11pt]{article} % use larger type; default would be 10pt
\usepackage{amsmath}
\usepackage{tikz}
\usepackage{geometry} % to change the page dimensions
\geometry{a4paper} % or letterpaper (US) or a5paper or....
\geometry{margin=1in}
\usepackage{circuitikz}

\usepackage{amsmath}
\usepackage{amsfonts}
\usepackage{amssymb}

\usepackage{amsthm}

\theoremstyle{definition}
\newtheorem{example}{Example}

\usepackage{fancyhdr} % This should be set AFTER setting up the page geometry
\pagestyle{fancy} % options: empty , plain , fancy
\renewcommand{\headrulewidth}{0pt} % customise the layout...
\lhead{ES3C5}\chead{Lecture 2}\rhead{09/01/2013}
\lfoot{}\cfoot{\thepage}\rfoot{Olly Levett}

\begin{document}

\section{Laplace transfer function (TF)}
\begin{itemize}
\item For a linear and stationary system
\begin{equation}
TF = \frac{\mathcal{L}\{output\}}{\mathcal{L}\{input\}}
\end{equation}
ie $\mathcal{L}\{output\} = TF x \mathcal{L}\{input\}$

with all intitial conditions assymed zero.
\item TF describes the dynamics of the system

\end{itemize}
A linear system obeys the principle of superposition, i.e. if $x_1 \rightarrow y_1$ and $x_2\rightarrow y_2$ then $x_1+x_2 \rightarrow y_1 + y_2$ where $x_n$ and $y_n$ are respectively the input and output of the sytem.

\subsection{Resistors}
\begin{figure}[h]

\begin{circuitikz}
	\draw
	(0,2) 
	to[short,o-*] (2,2) 
	node[anchor=west,above right]{$i(t)$}
	to[R, l=$R$,*-*] (2,0)
	to[short,*-o] (0,0);
	\draw[->] (0,0.3) --(0,1) node[left]{$v(t)$} --  (0,1.7);
\end{circuitikz}
\centering
\caption{Simple resistor network}
\end{figure}

\begin{equation}
v(t) = Ri(t)
\end{equation}

taking LT and assume zero initial conditions 
\begin{equation}
V(s) =RI(s)
\end{equation}
\subsection{Capacitors}

\begin{figure}[h]
\begin{circuitikz}
	\draw
	(0,2)
	to[short,o-*] (2,2) 
	node[anchor=west,above right]{$i(t)$}
	to[C, l=$C$,*-*] (2,0)
	to[short,*-o] (0,0);
	\draw[->] (0,0.3) --(0,1) node[left]{$v(t)$} --  (0,1.7);

\end{circuitikz}
\centering
\caption{Simple capacitor network}
\end{figure}
\begin{equation}
v(t) = \frac{1}{C}\int i(t) dt
\end{equation}
Taking LT and assume zero initial conditions
\begin{equation}
V(s) = \frac{I(s)}{sC}
\end{equation}

\subsection{Inductors}

\begin{figure}[h]
\begin{circuitikz}
	\draw
	(0,2)
	to[short,o-*] (2,2) 
	node[anchor=west,above right]{$i(t)$}
	to[L, l=$L$,*-*] (2,0)
	to[short,*-o] (0,0);
	\draw[->] (0,0.3) --(0,1) node[left]{$v(t)$} --  (0,1.7);
\end{circuitikz}
\centering
\caption{Simple inductor network}
\end{figure}
\begin{equation}
v(t) = L\frac{di(t)}{dt}
\end{equation}
Take LT and assume zero initial conditions
\begin{equation}
V(s) = sLI(s)
\end{equation}

\subsection{Kirchoff's Laws}
\begin{enumerate}
\item The total current flowing towards a node is equal to the total current flowing from that node
\item In a closed circuit, the algebraic sum of the products of the current and the resistance of each part of the circuit is equal to the resultant e.m.f. in the circuit. 

Alternatively, in a given loop, the sum of voltage rises is equal to the sum of voltage drops.

\end{enumerate}
The TF of a system can be found by finding the LT of each componenent and applying Kirchoff's laws.

\begin{example}
	See Fig. \ref{ex:int}
	\begin{figure}[h]
		\begin{circuitikz}
			\draw 	(0,2)
			to[R,l=$R$,o-*] (3,2) 
			to[C, l=$C$,*-*] (3,0)
			to[short,*-o] (0,0);

			\draw (3,2) to[short,*-o ](5,2);
			\draw (3,0) to[short,*-o] (5,0);

			\draw[->] (0,0.3) --(0,1) node[left]{$v_i(t)$} --  (0,1.7);
			\draw[->] (5,0.3) --(5,1) node[right]{$v_o(t)$} --  (5,1.7);
		\end{circuitikz}
		\centering
		\caption{An integrating circuit}
		\label{ex:int}
	\end{figure}

	\begin{equation}
		V_0(s)=\frac{I(s)}{sC}
	\end{equation}
From 2nd law,
	\begin{equation}
		V_i(s)=RI(s)+\frac{I(s)}{sC}
	\end{equation}
TF
	\begin{eqnarray}
		\frac{V_o}{V_i}&=&\frac{\frac{I(s)}{sC}}{RI(s)+\frac{I(s)}{sC}} \nonumber\\
		&=& \frac{1}{sRC+1}
	\end{eqnarray}
\end{example}
\begin{example}
	See Fig. \ref{ex:electnet}
	\begin{figure}[h]
		\begin{circuitikz}
			\draw 	(0,2)
			to[R,l=$R_1$,o-*] (3,2) 
			to[C, l=$C$,*-*] (3,0)
			to[short,*-o] (0,0);
			\draw (3,2) 
			to[L,*-*] (6,2)
			to[R, l=$R_2$,*-*] (6,0)
			to[short,*-*] (3,0);

			\draw (6,2) to[short,*-o ](8,2);
			\draw (6,0) to[short,*-o] (8,0);


			\draw[<-] (2.5,0.3) --(2.5,1) node[left]{$i_1(t)$} --  (2.5,1.7);
			\draw[<-] (5.5,0.3) --(5.5,1) node[left]{$i_2(t)$} --  (5.5,1.7);

			\draw[->] (0,0.3) --(0,1) node[left]{$v_i(t)$} --  (0,1.7);
			\draw[->] (8,0.3) --(8,1) node[right]{$v_o(t)$} --  (8,1.7);
		\end{circuitikz}
		\centering
		\caption{An example electrical network}
		\label{ex:electnet}
	\end{figure}

Applying 2nd law to 1st loop,
\begin{equation}
V_i(s)=R_1I_1(s)+\frac{I_1(s)}{sC}-\frac{I_2(s)}{sC}
\end{equation}
Applying 2nd law to 2nd loop,
\begin{equation}
\frac{I_2(s)}{sC} - \frac{I_1(s)}{sC} + sLI_2(s)+R_2I_2(s)=0
\end{equation}
\end{example} 
Solving simultaneously,
\begin{equation}
V_i(s)=I_2(s)\left( s^2LCR_1+s\left[ CR_1R_2+L\right] + \left[R_1+R_2\right]\right)
\end{equation}
note
\begin{equation}
V_o(S)=R_2I_2(s)
\end{equation}
$\therefore$ TF is
\begin{equation}
\frac{V_o}{V_i} = \frac{R_2}{s^2LCR_1+s\left[ CR_1R_2+L\right] + \left[R_1+R_2\right]}
\end{equation}

\section{Test signals and dynamic response}
\subsection{Unit step input}
	\begin{figure}[h]
		\begin{tikzpicture}
			\draw (0,0) -- (1,0) node[below]{$0$} -- (1,1) node [left] {$1$} -- (2,1) -- (4,1) node[right]{$x(t)$};
			\draw[->] (0,0) -- (4,0) node[below right]{$x$};
		\end{tikzpicture}
		\centering
		\caption{Step response}
	\end{figure}
\begin{equation}
X(s)=\frac{1}{s}
\end{equation}
\begin{example}
Given $H(s)=\frac{1}{1+\tau s}$ (a servo)
\begin{eqnarray}
Y(s) &=& \frac{1}{1+\tau s} \times \frac{1}{s} \nonumber \\
&=& \frac{\frac{1}{\tau}}{s\left( s+ \frac{1}{\tau}\right)}
\end{eqnarray}

\begin{figure}[h]
	\centering
	\caption{Dynamic response of a servo when subject to a unit step}
\end{figure}

\end{example}


\subsection{Unit ramp input}
\begin{figure}[h]
	\centering
	\caption{Dynamic response of a servo when subject to a unit step}
\end{figure}
Where $\tan (\theta) = 1$ ie
\begin{eqnarray}
x(t) &=& t \nonumber \\
X(s) = \frac{1}{s^2}
\end{eqnarray}

\begin{example}

Given $H(s) = \frac{1}{1+\tau s}$
\begin{eqnarray}
Y(s) &=& \frac{1}{1+\tau s} \times \frac{1}{s^2} \\
&=& \frac{\tau}{s+\frac{1}{\tau} - \frac{\tau}{s} + \frac{1}{s^2}}
\end{eqnarray}

Thus,
\begin{eqnarray}
y(t) &=& \tau e^{-\frac{t}{\tau}} - \tau + t \\
= t-\tau\left(1-e^{-\frac{t}{\tau}}\right)
\end{eqnarray}
\end{example}
\end{document}
